% This is part of the TFTB Reference Manual.
% Copyright (C) 1996 CNRS (France) and Rice University (US).
% See the file refguide.tex for copying conditions.



\markright{anaask}
\section*{\hspace*{-1.6cm} anaask}

\vspace*{-.4cm}
\hspace*{-1.6cm}\rule[0in]{16.5cm}{.02cm}
\vspace*{.2cm}



{\bf \large \sf Purpose}\\
\hspace*{1.5cm}
\begin{minipage}[t]{13.5cm}
Amplitude Shift Keyed (ASK) signal.
\end{minipage}
\vspace*{.5cm}


{\bf \large \sf Synopsis}\\
\hspace*{1.5cm}
\begin{minipage}[t]{13.5cm}
\begin{verbatim}
[y,am] = anaask(N)
[y,am] = anaask(N,ncomp)
[y,am] = anaask(N,ncomp,f0)
\end{verbatim}
\end{minipage}
\vspace*{.5cm}


{\bf \large \sf Description}\\
\hspace*{1.5cm}
\begin{minipage}[t]{13.5cm}
        {\ty anaask} returns a complex amplitude modulated signal of
        normalized frequency {\ty f0}, with a uniformly distributed random
        amplitude.  Such signal is only 'quasi'-analytic.\\

\hspace*{-.5cm}\begin{tabular*}{14cm}{p{1.5cm} p{8.5cm} c}
Name & Description & Default value\\
\hline
        {\ty N }    & number of points\\
        {\ty ncomp} & number of points of each component & {\ty N/5}\\
        {\ty f0}    & normalized frequency               & {\ty 0.25}\\
  \hline {\ty y}     & signal\\
        {\ty am }   & resulting amplitude modulation     \\
\hline
\end{tabular*}

\end{minipage}
\vspace*{1cm}


{\bf \large \sf Example}
\begin{verbatim}
         [signal,am]=anaask(512,64,0.05); 
         subplot(211); plot(real(signal)); 
         subplot(212); plot(am);
\end{verbatim}
\vspace*{.5cm}


{\bf \large \sf See Also}\\
\hspace*{1.5cm}
\begin{minipage}[t]{13.5cm}
\begin{verbatim}
anafsk, anabpsk, anaqpsk.
\end{verbatim}
\end{minipage}
\vspace*{.5cm}


{\bf \large \sf Reference}\\
\hspace*{1.5cm}
\begin{minipage}[t]{13.5cm}
[1] W. Gardner {\it Statistical Spectral Analysis - A Nonprobabilistic
Theory} Englewood Cliffs, N.J. Prentice Hall, 1987.
\end{minipage}
